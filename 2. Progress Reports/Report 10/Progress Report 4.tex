\documentclass[11pt]{report}

\usepackage{setspace}
%\setstretch{2.5} % for custom spacing
\setlength{\parindent}{4em}

\usepackage{fancyvrb}
\usepackage{graphicx}
\usepackage{geometry}
\usepackage{array}
\usepackage{hyperref}

\geometry{letterpaper, portrait, margin=1in}

%%%Title Page%%%
\title{
	\begin{center}
		\includegraphics[scale=0.5]{uga.PNG}\\
 	\end{center}
 	Progress Report 4
\bigbreak Group 23: Respiratory GPS Tracker
}

\author{\textbf{Zachary Davis}}

\date{\today}
%%%%%%%%%%%%

\begin{document}
\maketitle

\section*{Progress Report}
	\begin{center}
		\author{
		{\normalsize
		\begin{tabular}{m{4cm} m{2cm} m{10cm}}
		\textbf{Date} & \textbf{Hours} & \textbf{Description}\\
		\hline
		March 21st, 2018 & $3$ & We have completed the process of uploading data to our google cloud so that the positioning and timing data can be retrieved by the application.\\
		March 23rd, 2018 & $2$ & Worked on designing a case for the raspberry pi zero W, pozyx shield, and power supply so what will be attched to the various hospital equipment will be just one small unit.\\
		March 23rd - 30th, 2018 & $10$ & We spent most of our time since spring break working on\/developing the application as it is the final major milestone before we can put our whole project together. Focused on keeping the page depth and parallels as small as possible with location and equipment details. We are connected it to the DB with all the location data and preparing to calibrate for demonstration.\\
		March 25th-27th, 2018 & $2$ & Various work on the final connections of everything that we need in the final unit. Included removing default GPIO pins and soldering in on the required pins a specified length and trying to keep everything as neat exchangable and clean as possible.\\
		\end{tabular}
		}
		}
	\end{center}
		\textbf{Total Hours Since Last Update: } 17\\
		\textbf{Total Hours: } 95

\section*{Comments}
	\paragraph*{}
		These last two weeks have been about application development, database management, data transfer, and putting on the hardware together in a small neat package. We are still working on the application and design for an enclosure for everything in one unit, which will lead to final calibrating, testing, \& debugging.

\section*{Design Notebook Link}
	\begin{center}
		\href{https://docs.google.com/document/d/1_15R62LK1jZ8SYRuCrb-5IObISIdbLSfoH2bRU7464c/edit?usp=sharing}{Link}
	\end{center}
\end{document}