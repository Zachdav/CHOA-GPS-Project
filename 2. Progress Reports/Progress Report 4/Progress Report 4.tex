\documentclass[11pt]{report}

\usepackage{setspace}
%\setstretch{2.5} % for custom spacing
\setlength{\parindent}{4em}

\usepackage{fancyvrb}
\usepackage{graphicx}
\usepackage{geometry}
\usepackage{array}
\usepackage{hyperref}

\geometry{letterpaper, portrait, margin=1in}

%%%Title Page%%%
\title{
  Progress Report 3
\bigbreak Group 23: Respiratory GPS Tracker
}

\author{\textbf{Zachary Davis}}

\date{\bigskip
\today}
%%%%%%%%%%%%

\begin{document}
\maketitle

\section*{Progress Report}
	\begin{center}
		\author{
		{\normalsize
		\begin{tabular}{m{4cm} m{2cm} m{10cm}}
		\textbf{Date} & \textbf{Hours} & \textbf{Description}\\
		\hline
		October 18th - 26th, 2017 & 8 & Alex found a highly rated series of videos/tutorials on how to 
		program in Swift.  Swift is a language used to develop application in iOS the operating 
		system of all iPhones.  The iPhone application is what the clients requested to be the 
		user interface and the software used to communicate with the raspberry pies.  \\\\&&\href{https://www.youtube.com/watch?v=83WXmhin_LU}{Youtube Series}\\
		 & & \\
		October 20th - 23rd, 2017 & 2+ & Powering the pie is the only part of this project I am 
		currnetly up happy with.  Our clients did not want us to pull power from the equipment, 
		which is the most integrated, permenant, and clean solution.  The bandaid to the problem 
		is of course external batteries, which are ugly, unfinished, and consumable.  Instead, you 
		could uses and much larger battery power supply in the neighborhood of 10,000 mAh that 
		will last a long time and be rechargable. Thiw would look slightly better but nonetheless 
		is still a hassle for the end user that does not need to be there.  Especially in the case 
		of the nurses we spoke with who stressed many times that this has to help without adding 
		any steps to there current way of using the equipment.  Some how the idea of locating 
		equipment that now needs to be charged doesn't really do that.  I would like to reach out 
		to the client and see if the limitation of integrating power is set in stone and if it is 
		my final idea for now is solar.  A small panel on the outside of the enclosure that 
		provides the low amps needed by the pie as well as a pie that is in an off state when it 
		is obivous it wont be tracked.\\
		 & & \\
		October 23rd, 2017 & Under 1 & Alex began to look for enclosures for the raspberry 
		pie.  The reason we have waited so long to do this is because of a couple requirements.  
		We wanted to first determine whether or not we would need an external switch on the 
		enclosure for determining in-use/not-use state or a change in temperature from the 
		equipment.  We decided on a switch after now starting for two reasons.  One the change 
		in temperature method assumes that all equipment to be tracked have a large enough delta 
		and there is the added sensor expense on every tracker.  This all to say we now know we 
		need an enclosure that will incorperate a push-button.\\
		\end{tabular}
		}
		}
	\end{center}
		\textbf{Total Hours Since Last Update: } 10\\
		\textbf{Total Hours: } 35

\newpage

\section*{Comments}
	\paragraph*{}
		So far we have spent basically all of our time on the specifics of how to triangulate the 
		necessary hospital equipment and while we definitely have settled on a good idea it was 
		time to shift our focus.  Both of my group members and I began to learn how to program in
		Swift from tutorial videos found by Alex.  I too watched these videos and spent some time 
		in XCode practicing as well as translating our idea into software.  However another area 
		that needs some attention is power.  Dr. Johnsen asked a good question during our 
		presentation about how a battery is a fairly temporary answer to the problem.  However, 
		being that our clients do not want us to pull power from the equipment itself we are left 
		with a big problem.  Ontop of software I am spending time looking for a more convient and 
		permenant solution to the power problem.

\section*{Design Notebook Link}
	\begin{center}
		\href{https://docs.google.com/document/d/1_15R62LK1jZ8SYRuCrb-5IObISIdbLSfoH2bRU7464c/edit?usp=sharing}{Link}
	\end{center}
\end{document}